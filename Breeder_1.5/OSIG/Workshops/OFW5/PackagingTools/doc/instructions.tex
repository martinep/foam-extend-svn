\documentclass[11pt,a4paper,twocolumn]{article}
\usepackage[pdftex]{hyperref}
%\usepackage{english}
%\usepackage[latin1]{inputenc}
%\usepackage[T1]{fontenc}
%\usepackage{ae}
\usepackage[cm]{fullpage}
\usepackage[small,compact]{titlesec}
\usepackage{times}
\usepackage{tweaklist}
\usepackage{html}

\begin{document}

\renewcommand{\enumhook}{\setlength{\topsep}{0pt}%
  \setlength{\itemsep}{0pt}}

\setlength{\columnseprule}{1pt}

\makeatletter
\renewcommand{\@maketitle}{
\newpage
 \null
 \vskip 2em%
 \begin{center}%
  {\LARGE \@title \par}%
 \end{center}%
 \par} \makeatother
\title{Instructions for the 5th OpenFOAM Workshop USB-stick}
\maketitle

\section{What's on the USB-Stick?}

The USB-stick contains the workshop material (program, abstracts, presentations
and training material) which can be updated automatically via the Internet. In
addition it contains a fully operational Linux operating system based on
Kubuntu, and lots of useful CAE software including OpenFOAM (including pyFoam,
cgnstools funkySetFields, groovyBC and simpleFunctionObjects), Open CASCADE,
calculix, paraview with Takuya's OpenFOAM reader, enGrid, netgen, gmsh, blender,
freecad, qcad, elmer, gerris, xfoil, bladedesigner, mittel, skv, octave, yacas
and many more.

The workshop material is located in the OFW5 folder which resides in the
USB-stick's root-directory and can be accessed without booting the stick.  If
you boot from the USB-stick, you will find a link labeled \emph{5th Workshop} in
the \emph{Desktop Folder} which takes you there -- ``cd /cdrom/OFW5'' for those
who dislike GUIs.

\section{How to boot it?}

On ``standard'' laptops and PCs, things should be straight forward. Simply plug
the USB-stick in and boot your machine. You will be asked for the language you
prefer before you get to the kubuntu splash screen. Here select ``Try Kubuntu
without installing''. Booting your machine from the USB-stick will {\bf not}
change your normal operating system in any way. Later, if you like, you have the
option of installing Kubuntu and the included software on your computer
permanently.

If the machine boots from harddisk and {\bf not} into Kubuntu, you will have to
change the boot order in the BIOS.  You can enter the BIOS by pressing a
specific key during start-up (usually Del, F1, F2, F10, F12 - watch out for
instructions on the screen during the boot procedure).  In the BIOS, make sure
that the USB device is booted in preference to anything else, or choose a
one-time boot from the USB device.

\section{How to boot in a VirtualBox?}

This is for people who would like to run the USB-stick within their normal
operating system (Linux, Mac OS X, Windows). Here, we assume that the
VirtualBox-software is installed on your computer and that you are familiar with
the command line (the "Terminal Window") or at least know where to find it. You
will also need to have administrative rights on your machine, ie. you are
allowed to do ``sudo somecommand''.
\begin{latexonly}
The full instructions can be found here:
\href{http://web.student.chalmers.se/groups/ofw5/Instr.htm}{http://web.student.chalmers.se/groups/ofw5/Instr.htm}.
\end{latexonly}
\begin{htmlonly}
The general procedure is as follows:
%
\begin{enumerate}
\item Plug in the USB-stick
\item Find out which device the USB-stick is
\item Create a raw disk file using ``VBoxManage internalcommands
  createrawvmdk''
\item Create a new virtual machine in VirtualBox and attach the raw disk file
  that you just created
\item Boot the virtual machine. It should boot from the USB-stick
\end{enumerate}
%
{\bf Caution:} The command line parameters are only examples and have to be
adapted to your system. When doing this on a Mac you won't be able to access the
filesystem on the USB-stick from the Finder.
\end{htmlonly}

\begin{htmlonly}

{\bf on Linux:}
\begin{verbatim}
df
VBoxManage internalcommands createrawvmdk -filename ~/.VirtualBox/HardDisks/usb.vmdk -rawdisk /dev/sdb -register
# Sometimes you need to give special permissions ...
sudo chmod 666 /dev/sdb*
sudo chown felix ~/.VirtualBox/HardDisks/usb.vdmk
\end{verbatim}

{\bf on Mac OS X:}
\begin{verbatim}
df
# Unmount the USB-stick
sudo diskutil unmount /dev/disk2s1
VBoxManage internalcommands createrawvmdk -filename /Users/igarten/Desktop/stickBoot.vmdk -rawdisk /dev/disk2 -register
# The VirtualBox may complain that it cannot gain exclusive access to the disk 
# if the Finder mounted it again automatically. Redo the last two steps!
\end{verbatim}

{\bf on Windows:}
\begin{verbatim}
diskmgmt.msc
cd %programfiles%\sun\virtualbox
VBoxManage internalcommands createrawvmdk -filename "%USERPROFILE%"\.VirtualBox\VDI\usb.vmdk -rawdisk \\.\PhysicalDrive1 -register
\end{verbatim}

\end{htmlonly}

\section{How to alter the keyboard layout?}

The keyboard layout can be changed by clicking on the little US flag icon
located at the bottom right of the desktop. More keyboard layouts can be access
by clicking the right mouse button on it.

\section{How to use the WLAN at Chalmers?}

You should use the {\bf NOMAD} wireless network during the workshop. This network
is preconfigured and should come up automatically when you boot the USB-stick. The
control can be found at the bottom right of the desktop.

The network is not secured, but you must login via a web browser in order to use
it.  With your registration package you should have received instructions for
\emph{Internet Access at Chalmers}, and your personal username and password.

\section{How to update the workshop material?}

The workshop material (program, abstracts, presentations and training material)
can be updated via the Internet. If you boot from the USB-stick, you will find a
link labeled \emph{Update Workshop material} in the \emph{Desktop Folder} which
does the job. If your machine is booted into Linux or Mac OS X, you can use the
update script (cd OFW5; sudo ./update.sh) manually. Note that you must start the
script within the OFW5 directory. Sometimes the script complains that the stick
is mounted read-only. This can be fixed by issuing
{\small
\begin{verbatim}
sudo mount -o remount,rw /cdrom
\end{verbatim}}

There is no update functionality for Windows, but the material is available
online from
\begin{latexonly}
\href{http://web.student.chalmers.se/groups/ofw5/Program.htm}{http://web.student.chalmers.se/groups/ofw5/Program.htm}
\end{latexonly}
\begin{htmlonly}
\htmladdnormallink {http://web.student.chalmers.se/groups/ofw5/Program.htm} {http://web.student.chalmers.se/groups/ofw5/Program.htm}
\end{htmlonly}

\section{How to erase or resize the persistent storage?}

The persistent storage is used to save your result on the USB-stick. Currently,
512 Mb are allocated in casper-rw for this purpose. In case you want to reset
the persistent storage, simply do
{\small
\begin{verbatim}
dd if=/dev/zero of=casper-rw bs=1M count=512
mkfs.ext3 -F casper-rw
\end{verbatim}}

You may also resize {\bf without} loosing your data
{\small
\begin{verbatim}
resize2fs casper-rw 1024M
\end{verbatim}}


\end{document}
